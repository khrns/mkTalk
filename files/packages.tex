% Matehmatical symbols, letters and extras.
% See also:
% http://ctan.org/pkg/amsmath
% http://ctan.org/pkg/amsfonts
% http://www.ctan.org/tex-archive/macros/latex/contrib/bbm
\usepackage{amsmath, amsfonts, amssymb, bbm}

% Insert ancient Greek text coded in Beta Code.
% http://www.ctan.org/tex-archive/macros/latex/contrib/betababel
% (Extension of 'babel' (multilingual support for Plain TeX or LaTeX).
% http://ctan.org/pkg/babel)
\usepackage[english, greek]{betababel}

% A new bookmark (outline) organization for hyperref.
% https://www.ctan.org/pkg/bookmark
\usepackage{bookmark}

% Flexible handling of verbatim text including.
% http://www.ctan.org/pkg/fancyvrb
\usepackage{fancyvrb}

% Improved interface for floating objects.
% http://www.ctan.org/tex-archive/macros/latex/contrib/float
\usepackage{float}

% Input encoding to be used (for German umlauts etc.).
\usepackage[utf8]{inputenc}

% Longtable to write tables that continue to the next page and more.
% http://www.ctan.org/pkg/longtable
\usepackage{longtable}

% Highly configurable displaying of number and SI units.
% http://ctan.org/pkg/siunitx
\usepackage[%
   per-mode=symbol,                % When using \per, use a slash as delimiter.
   exponent-product=\cdot,         % When using 1eX notation, use \cdot as
                                   % delimiter.
   unit-optional-argument=true,    % Activate the optional argument usage for
                                   % units, e.g.: \metre[10] -> 10 m
   free-standing-units=true,       % The free-standing-units option controls
                                   % whether the unit macros exist outside of
                                   % the \si and \SI arguments.
   use-xspace=true,                % Well, use xspace to set the space (after a
                                   % unit) right.
   number-unit-product = \text{~}, % Put a protected space between number and
                                   % unit.
   separate-uncertainty=true,      % Print a given uncertainty as a separate
                                   % number (i.e. 3+-1 instead of 3(1)).
   ]{siunitx}

% Package to make slashed letters.
\usepackage{slashed}

% Automatic (correct) spacing for macros.
% http://ctan.org/pkg/xspace
\usepackage{xspace}
   \xspaceaddexceptions{]\}}  % Do the brackets right.

% Verbatim with URL-sensitive line breaks.
% http://www.ctan.org/pkg/url
\usepackage{url}

% NON-italic Greek letters.
\usepackage{upgreek}

% Use Latin modern fonts:
% (If you don't use it, you might need:
%    \RequirePackage{fix-cm}
% in front of \documentclass{...} to fix issues with Computer Modern fonts.)
% http://www.ctan.org/tex-archive/info/lmodern
\usepackage{lmodern}

% Package to place pictures side by side.
% http://www.ctan.org/pkg/subfig
\usepackage[
   caption=false, % If true, use the given settings for the 'caption' package.
                  % If 'caption' package is loaded, use the settings given
                  % there.
   ]{subfig}

% Create tabular cells spanning multiple rows.
% 'multicolumn' is part of 'longtable'.
% http://www.ctan.org/pkg/multirow
\usepackage{multirow}

% Intermix single and multiple columns.
% https://www.ctan.org/pkg/multicol
\usepackage{multicol}

% https://www.ctan.org/pkg/pbox
% A variable-width \parbox command.
\usepackage{pbox}

\renewcommand{\arraystretch}{1.1}                 %bigger line break in tables

%\usepackage[symbol,bottom,flushmargin]{footmisc}  %adapt your footer
%\usepackage[utf8x]{inputenc}                      %german umlauts etc.
%\usepackage{ucs}                                  %unicode in latex
